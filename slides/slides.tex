\documentclass{beamer}
\usepackage{listings}
\usepackage{graphicx}

\setbeamertemplate{headline}{}

\lstset{
frame=single
}

\begin{document}

\title{Writing API Services in Go}
\author{Nandakumar Edamana}

\maketitle

\begin{frame}[fragile]{Question}
What do you mean by API Server Software? Think about a single program that
serves an API. How does it differ from a program that accepts two numbers
via the command line, prints the sum and quits?
\end{frame}

\begin{frame}[fragile]{API Server Characteristics}
\begin{itemize}
  \pause \item Daemon
  \pause \item Serves API over HTTP over TCP
  \pause \item Sticks to something like REST (quality of the API rather than
               the program itself)
\end{itemize}
\end{frame}

\begin{frame}[fragile]{Levels}
\begin{itemize}
  \item HTTP server in Go using no third-party libraries
  \item HTTP server in Go using a third-party router (chi)
  \item Auto-generation of boilerplate using OpenAPI
  \item Writing multiple services and combining them using an API Gateway
\end{itemize}

NOTE: OpenAPI is for more than mere code generation.

\end{frame}

\begin{frame}[fragile]{Packages}
\begin{itemize}
  \item net/http - Built-in library for HTTP client, server and mux; supports middlewares, does not support RegEx path matching.
  \item chi, gorilla/mux, etc. - Feature-rich HTTP routers
\end{itemize}
\end{frame}

\begin{frame}[fragile]{Hello World}
\begin{lstlisting}
mkdir 1-hello-world && cd 1-hello-world
go mod init helloapi
# "to add module requirements and sums"
go mod tidy
\end{lstlisting}
\end{frame}

\begin{frame}[fragile]{Hello World}
\begin{lstlisting}[language=Go]
package main

import (
  "fmt"
  "log"
  "net/http"
)
\end{lstlisting}
\end{frame}

\begin{frame}[fragile]{Hello World}
\begin{lstlisting}[language=Go]
func main() {
  // Adds to DefaultServeMux
  // Note: you can use anonymous functions
  http.HandleFunc("/", handleIndex)
  http.HandleFunc("/hello", handleHello)
  
  fmt.Println("Listening at :8080...")
  
  // Address and handler;
  // handler = nil means DefaultServeMux
  err := http.ListenAndServe(":8080", nil)
  if err != nil {
    log.Fatal(err)
  }
}
\end{lstlisting}
\end{frame}

\begin{frame}[fragile]{Hello World}
\begin{lstlisting}[language=Go]
func handleHello(w http.ResponseWriter,
                 r *http.Request) {

  w.Write([]byte("world!\n"))
}

func handleIndex(w http.ResponseWriter,
                 r *http.Request) {

  w.Write([]byte("Welcome!\n"))
}
\end{lstlisting}
\end{frame}

\begin{frame}[fragile]{Hello World}
\begin{lstlisting}
$ go run .
Listening at :8080...
\end{lstlisting}
\end{frame}

\begin{frame}[fragile]{Hello World}
\begin{lstlisting}
$ curl localhost:8080/
Welcome!

$ curl localhost:8080/hello
world!
\end{lstlisting}
\end{frame}

\begin{frame}[fragile]{Hello World}
\begin{lstlisting}
$ go run .
Listening at :8080...
^Csignal: interrupt
\end{lstlisting}
\end{frame}

\begin{frame}{The Complexity Underneath}
See the manpage of bind(2) and add forked or threaded serving.
\end{frame}

\begin{frame}{Questions?}
\end{frame}

\begin{frame}{Thank You}
\end{frame}

\end{document}
